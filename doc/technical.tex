%xelatex -shell-escape -output-directory=bin ergasia.tex
\documentclass{assignment}

\university{Πανεπιστήμιο Πειραιώς}{Πα.Πει.}
\school{Τμήμα Πληροφορικής}{Π.Μ.Σ. "Πληροφορική"}
\department{Πρόγραμμα Μεταπτυχιακών Σπουδών «Πληροφορική»}{}
%\cover{images/cover.jpg}{http://www.cyberciti.biz/faq/grub-boot-into-single-user-mode/}

\title{Τεχνολογίες Διαδικτύου \\ 2η Εργασία}
%\projectlevel{Εργαστήριο Λειτουργικά Συστήματα}
%\lesson{Λειτουργικά Συστήματα}{1}
\date{Αθήνα, 2014}

\author{Αναγνωστόπουλος Βασίλης - Θάνος, Κατσής Γεώργιος}
%\register{ΜΠΠΛ13002}{1}

%\exercauthor{Αναγνωστόπουλος Βασίλης - Θάνος}{06107083}{9}

%\advisor{Τσακίρη Μαρία, Αναπληρώτρια Καθηγήτρια Ε.Μ.Π.}

\begin{document}

\maketitle
% Να σκεφτώ τί αλλαγές θέλω να κάνω με τις αριθμήσεις και άμα θέλω να κάνω.
% Να σκεφτώ να τις ενσωματώσω και στο assignment.cls

\setcounter{page}{1} 
\pagenumbering{roman}

\pagestyle{plain}
\tableofcontents
\newpage


%\pagestyle{headings}
\pagestyle{fancy}
\setcounter{page}{1} 
\pagenumbering{arabic}

\section{Ανάλυση απαιτήσεων}

Η δημιουργία του συστήματος διεπαφής με τους χρήστες ενός θεματικού πάρκου μιας μεσαιωνικής καστροπολιτείας, η οποία θα αξιοποιηθεί τουριστικά με την προσθήκη ηλεκτρονικού αυτοματισμού, με σκοπό την προσέλκυση τουρισμού.

Στην μεσαιωνική καστροπολιτεία, θα διατίθενται ανεξάρτητα μεσαιωνικά διαμερίσματα, καθένα από τα οποία θα περιβάλλεται από μία τάφρο. Το κάθε διαμέρισμα θα έχει μία πόρτα που θα ανεβαίνει και θα κατεβαίνει και θα λειτουργεί ως γέφυρα πάνω στην τάφρο. Η τάφρος δεν θα υπάρχει μόνο για λόγους ασφάλειας αλλά και για να δημιουργεί ατμόσφαιρα διασκέδασης οπότε θα λειτουργεί επίσης ως ιδιωτική πισίνα για κάθε μεσαιωνικό διαμέρισμα.

Επίσης, στην καστροπολιτεία, θα λειτουργεί καφετέρια-εστιατόριο και σε αυτό το πλαίσιο, οι λειτουργίες κάποιων μηχανημάτων μπορούν να γίνονται ηλεκτρονικά. Θα περιλαμβάνονται κάποιες εικονικές διεπαφές με τους χρήστες για τομείς που δεν είναι ανάγκη να είναι υλοποιήσιμοι (όμως για τον χρήστη θα είναι λειτουργικοί, π.χ. ο χρήστης θα μπορεί να πληρώσει με πιστωτική κάρτα, απλά όπως είναι κατανοητό δεν θα είναι αληθινή η συναλλαγή).  

Συγκεκριμένα ζητούνται τα παρακάτω:

Λειτουργικότητα της εφαρμογής

Θεωρείστε ότι οι χρήστες της εφαρμογής αποτελούνται από τους πελάτες-τουρίστες της μεσαιωνικής καστροπολιτείας. Επίσης στην εφαρμογή θα υπάρχουν και οι υπάλληλοι της μεσαιωνικής καστροπολιτείας. Οι χρήστες μέσω της εφαρμογής θα μπορούν να κάνουν τα εξής:

1)Προσομοίωση αλληλεπίδρασης με τις διάφορες συσκευές. Θα υποθέσετε ότι είναι δυνατόν να δημιουργήσετε σύστημα διεπαφής για την διαχείριση (από τους πελάτες ή τους υπαλλήλους) των διαφόρων συσκευών (ηλεκτρικών και άλλων) του κάθε μεσαιωνικού διαμερίσματος μέσω υπολογιστή και θα φτιάξετε το σύστημα διεπαφής για αυτή την διαχείριση. Για παράδειγμα, μπορείτε να υποθέσετε ότι μπορούν να ανάψουν ή να σβήσουν τα φώτα του μεσαιωνικού διαμερίσματος μέσω υπολογιστή, ότι μπορεί να ανάψουν, να σβήσουν και να ρυθμιστούν ή θέρμανση ή η ψύξη μέσω υπολογιστή, να ανοίξει ή να κλείσει η τηλεόραση ή το ραδιόφωνο μέσω υπολογιστή, όλες οι ηλεκτρικές συσκευές που έχει το διαμέρισμα κ.λ.π. . Να σχεδιάσετε τον τρόπο με τον οποίο θα μπορούν να δοθούν αυτές οι εντολές στο μεσαιωνικό διαμέρισμα από τον υπολογιστή.

2)Προσομοίωση ηλεκτρονικής διαχείρισης της τάφρου-πισίνας και της πόρτας που ανεβοκατεβαίνει ώστε να γίνει γέφυρα πάνω στην τάφρο-πισίνα. Ας υποθέσετε ότι σε μία κονσόλα, οι πελάτες ή υπάλληλοι της καστροπολιτείας μπορούν να ενημερωθούν για την κατάσταση της τάφρου-πισίνας για την πόρτα-γέφυρα και να δώσουν εντολές σχετικά με αυτές.

Εντολές τάφρου-πισίνας: Θα υποθέσετε ότι είναι δυνατό να δημιουργήσετε συστήματα διεπαφής όπου οι πελάτες του μεσαιωνικού διαμερίσματος ή οι υπάλληλοι της καστροπολιτείας θα μπορούν να δίνουν εντολές σε μηχανισμού που θα αδειάζουν ή θα γεμίζουν την τάφρο με νερό σε διάφορες στάθμες, θα ζεσταίνουν ή θα κρυώνουν το νερό σε διάφορα επίπεδα θερμοκρασίας. Επίσης, θα υπάρχει αισθητήρας που θα ενεργοποιείται ή θα απενεργοποιείται και θα ειδοποιεί τον χρήστη εάν υπάρχουν άνθρωποι μέσα στην πισίνα. Θα υπάρχει η δυνατότητα να ενεργοποιείται συναγερμός από αυτόν τον αισθητήρα ανάλογα την περίπτωση (π.χ. είναι βράδυ και ο ένοικος θέλει να υπάρχει συναγερμός για λόγους ασφαλείας, ή είναι μέρα και ο ένοικος δεν θέλει το συναγερμό λόγω του ότι η τάφρος λειτουργεί ως πισίνα). Οι εντολές τάφρου-πισίνας θα πρέπει να ολοκληρωθούν και να συμπληρωθούν σύμφωνα με την ανάλυση εργασιών που θα διεξαχθεί.

Εντολές πόρτας-γέφυρας. Οι εντολές τάφρου-πισίνας θα πρέπει να ολοκληρωθούν και να συμπληρωθούν σύμφωνα με την ανάλυση εργασιών που θα διεξαχθεί από σας.


Προσομοίωση αλληλεπίδρασης πελατών με τους υπαλλήλους της καστροπολιτείας. Θα υποθέσετε ότι είναι δυνατόν να δίνονται παραγγελίες από τους πελάτες με σκοπό την εξυπηρέτηση τους, καθώς και ερωτήσεις αποκρίσεις από τους υπαλλήλους της καστροπολιτείας για την εκκίνηση της παραγγελίας, την πραγματοποίηση και την ολοκλήρωση της. Αυτές οι παραγγελίες μπορεί να αφορούν καφέ ή/ και κάποιο γεύμα ή/και κάποιο ποτό, ανάλογα με την ώρα της ημέρας. Καλείστε να δημιουργήσετε σύστημα διεπαφής για αυτή την αλληλεπίδραση πελατών-υπαλλήλων, την διαδικασίας παραγγελίας, την εκτέλεση της και την εξόφληση της ή την ενσωμάτωση του λογαριασμού στο γενικό λογαριασμό του πελάτη. Ο υπάλληλος θα πρέπει να φαίνεται ως εικόνα ενός ιππότη ή μιας πριγκίπισσας. 

Γενική προσομοίωση του συστήματος διεπαφής με τον χρήστη-πελάτη. Ο σχεδιασμός θα πρέπει να έχει γίνει με έναν τρόπο τέτοιο ώστε να έρχεται όσο πιο κοντά στο θέμα της μεσαιωνικής καστροπολιτείας. Οπότε θα περιέχει στοιχεία όπως κατάλληλη διακόσμηση, πολεμίστρες, μπουντρούμια, κήπους με δέντρα στον εξωτερικό χώρο πριν μπούμε μέσα στο μεσαιωνικό διαμέρισμα, κιάλα, παράθυρα με θέα στην τάφρο κ.λ.π. 



\subsection{Ιεραρχική ανάλυση εργασιών}

Task analysis the analysis of how a task is accomplished, including a detailed description of both manual and mental activities, task and element durations, task frequency, task allocation, task complexity, environmental conditions, necessary clothing and equipment, and any other unique factors involved in or required for one or more people to perform a given task.[1] Task analysis emerged from research in applied behavior analysis and still has considerable research in that area. \cite{wiki:task_analysis}

\section{Σχεδιασμός}

\section{Υλοποίηση και επαλήθευση}

\newpage

\phantomsection \label{Βιβλιογραφία}
\addcontentsline{toc}{section}{Βιβλιογραφία}
%\mtcaddchapter[Βιβλιογραφία] % Λόγω του minitoc
\bibliographystyle{plain}
\bibliography{references}


\end{document}

