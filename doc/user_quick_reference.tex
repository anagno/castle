\documentclass[a4paper,titlepage,twoside,12pt,leqno]{article}

\usepackage[]{fontspec}
\usepackage{xltxtra}
\usepackage[monogreek]{xgreek}

\newcommand{\en}[1]{\setlanguage{american}#1\setlanguage{monogreek}} % Για υφενώσεις στα αγγλικά

\defaultfontfeatures{Mapping=tex-text%, Scale=MatchLowercase}
} 

% Γραμματοσειρά
\setmainfont[Mapping=tex-text]{DejaVu Sans}


\usepackage{longtable} % για έναν μεγάλο πίνακα
\usepackage{graphicx, array, blindtext}

%Για την εντολή \caption* και την αφαίρεση του πίνακας 1:
\usepackage{caption}

\title{Εγχειρίδιο γρήγορης αναφοράς για το ηλεκτρονικό θεματικό πάρκο σε μεσαιωνική καστροπολιτεία}
\author{Αναγνωστόπουλος Βασίλης-Θάνος, Κατσής Γιώργος}
\date{}

\begin{document}

\maketitle
\tableofcontents
\listoffigures
\listoftables
\newpage

\section{Εισαγωγή}

%Στα εγχειρίδια αυτά παρουσιάζεται η σύνταξη των εντολών με πολύ συνοπτικό τρόπο. Σε αυτά ανατρέχει ένας χρήστης ο οποίος γνωρίζει πως να χρησιμοποιεί ένα πρόγραμμα σε γενικές γραμμές αλλά δεν θυμάται τη σύνταξη κάποιας συγκεκριμένης εντολής.


\end{document}
