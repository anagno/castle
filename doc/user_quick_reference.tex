\documentclass[a4paper,titlepage,twoside,12pt,leqno]{article}

\usepackage[]{fontspec}
\usepackage{xltxtra}
\usepackage[monogreek]{xgreek}

\newcommand{\en}[1]{\setlanguage{american}#1\setlanguage{monogreek}} % Για υφενώσεις στα αγγλικά

\defaultfontfeatures{Mapping=tex-text%, Scale=MatchLowercase}
} 

% Γραμματοσειρά
\setmainfont[Mapping=tex-text]{DejaVu Sans}


\usepackage{longtable} % για έναν μεγάλο πίνακα
\usepackage{graphicx, array, blindtext}

%Για την εντολή \caption* και την αφαίρεση του πίνακας 1:
\usepackage{caption}

\title{Εγχειρίδιο γρήγορης αναφοράς για το ηλεκτρονικό θεματικό πάρκο σε μεσαιωνική καστροπολιτεία}
\author{Αναγνωστόπουλος Βασίλης-Θάνος, Κατσής Γιώργος}
\date{}

\begin{document}

\maketitle

%\toc Να μπουν περιεχόμενα

\section{Εισαγωγή}

Το συγκεκριμένο πρόγραμμα προορίζεται για την αλληλεπίδραση του ενοίκου του μεσαιωνικού κάστρου με τις διάφορες συσκευές που διαθέτει το διαμέρισμα του. Συγκεκριμένα θα μπορεί να διαχειριστεί τις παρακάτω συσκευές:

\begin{itemize}
\item Πισίνα (βλέπε σελ. \ref{})


\end{itemize}

\section{Πλοήγηση στην εφαρμογή}

Εδώ να βάλουμε φωτό από τα εικονίδια και τι κάνει το καθένα

\section{Να σκεφτούμε μην βάλουμε FAQ}

\end{document}
